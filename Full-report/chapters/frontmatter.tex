\maketitle
\makesignature

\ifproject
\begin{abstractTH}
% เขียนบทคัดย่อของโครงงานที่นี่

% การเขียนรายงานเป็นส่วนหนึ่งของการทำโครงงานวิศวกรรมคอมพิวเตอร์
% เพื่อทบทวนทฤษฎีที่เกี่ยวข้อง อธิบายขั้นตอนวิธีแก้ปัญหาเชิงวิศวกรรม และวิเคราะห์และสรุปผลการทดลองอุปกรณ์ และระบบต่าง ๆ
% \enskip อย่างไรก็ดี การสร้างรูปเล่มรายงานให้ถูกรูปแบบนั้นเป็นขั้นตอนที่ยุ่งยาก
% แม้ว่าจะมีต้นแบบสำหรับใช้ในโปรแกรม Microsoft Word แล้วก็ตาม
% แต่นักศึกษาส่วนใหญ่ยังคงค้นพบว่าการใช้งานมีความซับซ้อน และเกิดความผิดพลาดในการจัดรูปแบบ กำหนดเลขหัวข้อ และสร้างสารบัญอยู่
% \enskip ภาควิชาวิศวกรรมคอมพิวเตอร์จึงได้จัดทำต้นแบบรูปเล่มรายงานโดยใช้ระบบจัดเตรียมเอกสาร
% \LaTeX{} เพื่อช่วยให้นักศึกษาเขียนรายงานได้อย่างสะดวกและรวดเร็วมากยิ่งขึ้น

การระบุตัวตนด้วยการใช้รูปภาพใบหน้าบุคคลเพื่อเข้าสู่สถานที่เป็นหนึ่งในรูปแบบการยืนยันตัวตนที่ช่วยลดการแพร่ระบาดของโรคติดเชื้อไวรัสโคโรนา (COVID-19) โดยปัญหาหลักของการระบุตัวตนด้วยการใช้รูปภาพใบหน้าบุคคลคือการเปลี่ยนแปลงของใบหน้า 
เช่น ผมยาวขึ้น สวมแว่นตา สวมหน้ากากอนามัย และสวมหมวก เนื่องจากความแม่นยำในการระบุตัวตนจะต่ำเมื่อใบหน้าบุคคลมีการเปลี่ยนแปลง ทางเข้าสถานที่มีพื้นที่จำกัด และในสถานการณ์ปัจจุบันที่ราคาของอุปกรณ์อิเล็กทรอนิกส์มีราคาที่สูงขึ้น 
จึงได้ออกแบบมอดูลกล้องให้มีขนาดเล็ก ซึ่งประกอบไปด้วย กล้องเว็บแคม แผงแป้นพิเศษแบบตาราง ปุ่มกด และเลือกใช้ Raspberry Pi เป็นหน่วยประมวลผลกลางในมอดูลกล้อง แต่ประสิทธิภาพของ Raspberry Pi ไม่เพียงพอต่อการระบุตัวตนของรูปภาพใบหน้าบุคคล 
หรือ เรียนรู้รูปภาพใบหน้าบุคคล ดังนั้น Raspberry Pi จึงทำหน้าที่ตรวจจับใบหน้าบุคคล ส่งรูปภาพใบหน้าไปยังเซิร์ฟเวอร์ แสดงผลการระบุตัวตน และรับรหัสการยืนยันตัวตน หรือ ปุ่มกดยืนยันตัวตน และใช้เซิร์ฟเวอร์ในการระบุตัวตนด้วยรูปภาพใบหน้าบุคคล 
นำรูปภาพใบหน้าบุคคลที่ได้รับมาจากมอดูลกล้องในแต่ละครั้งทำการเรียนรู้รูปภาพใบหน้าบุคคลเป็นแบบจำลองรูปภาพใบหน้าบุคคลเพื่อปรับให้แบบจำลองรูปภาพใบหน้าบุคคลรองรับรูปภาพใบหน้าใหม่ของบุคคล และจัดเก็บรูปภาพใบหน้าบุคคล
โดยผลลัพธ์ของการระบุตัวตนมีแนวโน้วที่สูงขึ้น เวลาในการประมวลผลรูปภาพใบหน้าบุคคลตลอดถึงการแสดงผลอยู่ในระดับที่ผู้ใช้งานรับได้ และมอดูลกล้องมีขนาดเล็กสามารถติดตั้งที่ทางเข้าอาคาร


% จึงมีการใช้กลวิธีการเรียนรู้แบบต่อเนื่อง เพื่อนำรูปภาพใบหน้าใหม่ไปทำการเรียนรู้ 
% ซึ่งก็คือ 
% เพื่อลดจุดอ่อนของการใช้รูปภาพใบหน้าบุคคล โดยได้พัฒนาระบบและทำการทดลองที่ทางเข้าสถานที่นั้นมีพื้นที่ในการติดตั้งน้อยในโครงงานนี้จึงเลือกใช้ Raspberry Pi และในโปรเจกต์นี้ได้ทำการติดตั้งที่ทางเข้าห้องแลป OASYS 
% เนื่องจาก Raspberry Pi นั้นมีประสิทธิภาพไม่เพียงพอต่อการระบุตัวตน หรือ การเรียนรูปรูปภาพใบหน้าบุคคล ในโปรเจกต์นี้จึงใช้ Raspberry Pi ทำหน้าที่ตรวจจับใบหน้า และใช้เซิร์ฟเวอร์ในการระบุตัวตน 
% และเรียนรู้รูปภาพใบหน้าบุคคล โดยแบบจำลองการตรวจจับใบหน้าบน Raspberry Pi นั้นสามารถที่จะตรวจจับใบหน้าบุคคลได้ดี ซึ่งเวลาเฉลี่ยในการแสดงผลการระบุตัวตนหลังจากการตรวจจับใบหน้านั้นน้อยกว่า 500 มิลลิวินาที 
% และยังมีการยืนยันตัวตนด้วยรหัสอีกขึ้นตอนเพื่อลดความผิดพลาดในการระบุตัวตน ทำให้ผู้ใช้งานยังได้รับความสะดวก ปลอดภัยจากโรคติดเชื้อไวรัสโคโรนา (COVID-19) จากการใช้งาน และยังมีความแม่นยำ

\end{abstractTH}

\begin{abstract}
% The abstract would be placed here. It usually does not exceed 350 words
% long (not counting the heading), and must not take up more than one (1) page
% (even if fewer than 350 words long).

% Make sure your abstract sits inside the \texttt{abstract} environment.


Facial recognition using images of individuals is one of the forms of identity verification that helps reduce the spread of the coronavirus (COVID-19). 
The main problem with identifying individuals using facial images is the changes that occur on the face, such as when someone grows a beard, wears glasses, 
puts on a mask, or wears a hat. The accuracy of facial recognition decreases when facial features change. The entry way has a small space, 
and in the current situation where electronic equipment prices are high, a small-sized camera module has been designed. It includes a webcam, a special keyboard, buttons, 
and a Raspberry Pi used as a central processing unit. However, the performance of the Raspberry Pi is not sufficient for facial recognition of individuals or 
learning facial images. Therefore, the Raspberry Pi is responsible for detecting facial images, sending them to the server for recognition, 
receiving confirmation codes or buttons, and using the server to recognize facial images. Facial images from the camera module are learned each time, 
using a facial image model to adjust the model to support new facial images of individuals and store them. As a result, 
the accuracy of identity verification has improved, the processing time for facial images is at an acceptable level for users, 
and the camera module is small in size and can be used at entryways.




% Facial recognition technology is one form of identity verification that can help reduce the spread of COVID-19 by allowing individuals to 
% access buildings without the need for physical contact. However, using facial images has its weaknesses when a person's face changes, 
% such as by growing a beard or wearing glasses. To improve the accuracy of new facial images, adaptive machine learning algorithms are employed.

% Due to limited space, this project utilized Raspberry Pi, which was installed at the entrance of the OASYS lab. 
% Although Raspberry Pi's performance is insufficient for facial recognition and learning images, 
% it was used to detect faces while a server was used for recognizing and training facial models. 
% The facial detection model on the Raspberry Pi works well and takes less than 500 milliseconds on average to 
% verify identity after detecting a face. Additionally, to reduce the likelihood of identity errors, there is a pin code 
% confirmation process that makes it convenient and safe for users to use and helps prevent the spread of COVID-19.


\end{abstract}

\iffalse
\begin{dedication}
This document is dedicated to all Chiang Mai University students.

Dedication page is optional.
\end{dedication}
\fi % \iffalse

\begin{acknowledgments}
\indent โครงงานนี้จะไม่สำเร็จลุล่วงลงได้ ถ้าไม่ได้รับความกรุณาจาก ผศ.ดร.ภาสกร แช่มประเสริฐ
อาจารย์ที่ปรึกษา ที่ได้สละเวลาให้ความช่วยเหลือทั้งให้คำแนะนำ ให้ความรู้และแนวคิดต่าง ๆ รวมถึง
อ.ดร.\,ณัฐนันท์ พรหมสุข และ ผศ.ดร.กำพล วรดิษฐ์ ที่ให้คำปรึกษาจนทำให้โครงงานเล่มนี้เสร็จ
สมบูรณ์ไปได้ \\
\indent ขอบคุณห้องวิจัย OASYS ภาควิชาวิศวกรรมคอมพิวเตอร์ คณะวิศวกรรมศาสตร์
มหาวิทยาลัยเชียงใหม่ ที่เอื้อเฟื้อสถานที่ในการทำโครงงาน สนับสนุนอุปกรณ์ต่าง ๆ และ
ขอขอบคุณ นาย กมลพัฒน์ สุนทรพงศ์ และ นางสาวโชติโรส ประถม ที่คอยให้ความช่วยเหลือในการทำโครงงานมาโดยตลอด \\
\indent ขอขอบคุณทาง ITSC ที่ได้ให้ใช้เซิร์ฟเวอร์ และขอบคุณเพื่อน ๆ ที่ให้กำลังใจรวมถึงคำแนะนำที่ดีตลอดการทำโครงงานที่ผ่านมา 
รวมทั้งขอขอบพระคุณอีกหลาย ๆ ท่านที่ไม่ได้เอ่ยนามมา ณ ที่นี้ ที่ได้ให้ความ
ช่วยเหลือตลอดมา หากหนังสือโครงงานเล่มนี้มีข้อผิดพลาดประการใด กระผมขอน้อมรับด้วยความ
ยินดี


% \texttt{acknowledgment} environment.

\acksign{2022}{12}{12}
\end{acknowledgments}%
\fi % \ifproject

\contentspage

\ifproject
% \figurelistpage

% \tablelistpage
\fi % \ifproject

% \abbrlist % this page is optional

% \symlist % this page is optional

% \preface % this section is optional
