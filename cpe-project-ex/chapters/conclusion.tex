\chapter{\ifenglish Conclusions and Discussions\else บทสรุปและข้อเสนอแนะ\fi}

\section{\ifenglish Conclusions\else สรุปผล\fi}

ในการทำโครงงานระบบการยืนยันตัวตนด้วยการใช้ภาพใบหน้าเพื่อเข้าสู่สถานที่ด้วยการใช้กลวิธีการเรียนรู้ของเครื่องแบบต่อเนื่องสามารถพัตนาให้มอดูลตรวจจับใบหน้านั้นตรวจจับใบหน้าได้แม่นยำถึง 100 เปอร์เซ็นต์ 
เมื่อนำไปติดตั้งทดสอบสามารถทำงานได้อย่างถูกต้อง ใช้พื้นที่ติดตั้งน้อย และเวลาเฉลี่ยในการทำงานระหว่างเซิร์ฟเวอร์กับมอดูลกล้องคือ 455.618 มิลลิวินาที


\section{\ifenglish Challenges\else ปัญหาที่พบและแนวทางการแก้ไข\fi}

ในการทำโครงงานนี้ พบว่าเกิดปัญหาหลัก ๆ ดังนี้
\begin{enumerate}
    \item เมื่อรูปภาพใบหน้ามีความคมชัดน้อย จะทำให้แบบจำลองการตรวจจับใบหน้าไม่สามารถนำรูปภาพใบหน้าไปเรียนรู้รูปภาพใบหน้าเป็นแบบจำลองการรระบุตัวตน จึงต้องมีการใช้การปรับรูปภาพให้ดีขึ้น เช่น การลดสัญญาณรบกวนที่เกิดขึ้นในภาพ การเพิ่มความคมชัด 
    และการบีบอัดคงสัญญาณ เป็นต้น
    \item ในการตรวจจับรูปภาพใบหน้าบนมอดูลกล้อง เมื่อผู้ใช้งานขยับตัวในขณะที่กล้องกำลังโฟกัสใบหน้าบุคคล จะทำให้รูปภาพใบหน้าบุคคลไม่ชัดส่งผลให้ผลลัพธ์ของรูปภาพที่ส่งไประบุตัวตนมีความผิดพลาดสูง 
    จึงมีการกำหนดข้อตกลงให้ผู้ใช้อยู่นิ่งในตอนที่กล้องกำลังโฟกัส
    \item เมื่อมอดูลกล้องเปิดใช้งานเป็นเวลานานจะเกิดปัญหาหน่วยความจำชั่วคราวเต็ม ซึ่งจะต้องทำการเปิดโปรแกรมใหม่
    \item เมื่ออินเทอร์เน็ตมีปัญหา เช่น หลุด หรือ ไฟดับ จะทำให้โปรแกรมบนมอดูลกล้องนั้นค้างต้องมีการตรวจสอบอินเทอร์เน็ตทุกวัน
    \item การจ่ายไฟให้กับมอดูลกล้องจะต้องใช้ตัวปรับกระแสไฟที่มีกำลังการจ่ายไฟที่เพียงพอ เนื่องจากกำลังไฟที่น้อยไปจะทำให้ประสิทธิภาพของ Raspberry Pi ลดลง
    \item ในขั้นตอนการกดปุ่มกดตัวเลขเพื่อยืนยันตัวตน เมื่อผู้ใช้กดปุ่มค้างจะทำให้เลขยืนยันตัวตนจะผิดพลาดไปด้วย และเมื่อหน่วงเวลาของปุ่มกดมากไปจะทำให้ผู้ใช้กรอกรหัสการยืนยันตัวตนผิดพลาด
\end{enumerate}



\section{\ifenglish%
Suggestions and further improvements
\else%
ข้อเสนอแนะและแนวทางการพัฒนาต่อ
\fi
}

ข้อเสนอแนะเพื่อพัฒนาโครงงานนี้ต่อไป มีดังนี้
\begin{enumerate}
    \item สามารถที่จะพัฒนาแบบจำลองการเรียนรู้รูปภาพใบหน้า โดยใช้กลวิธีการเรียนรู้ของเครื่องในรูปแบบอื่น ๆ เพื่อเพิ่มความแม่นยำในการระบุตัวตนขณะที่ใส่หน้ากากอนามัยสูงขึ้น
    \item ในการออกแบบมอดูลกล้องสามารถออกแบบให้กล่องเก็บ Raspberry Pi นั้นมีขนาดเล็กลงก็จะใช้พื้นที่ลดลงได้
    \item ปุ่มกดยืนยันตัวตนสามารถที่จะเปลี่ยนไปใช้ปุ่มที่ใช้ระบบอินฟราเรด หรือ ระบบยืนยันแบบการชูจำนวนนิ้วเพื่อบ่งบอกตัวเลขผ่านกล้อง (Hand Gesture Recognition) ก็จะลดการสัมผัสทำให้ลดการแพร่กระจายของโรคติดต่อ 
    \item ควรพัฒนาแบบจำลองการตราจจับใบหน้าให้มีประสิทธิภาพมากขึ้น เนื่องจากเมื่อประสิทธิภาพมากขึ้นจะทำให้การตรวจจับใบหน้านั้นรวดเร็วยิ่งขึ้น
    \item ควรพัฒนาวิธีการเพิ่มความชัดของรูปภาพใบหน้า เนื่องจากมีผลต่อความแม่นยำในการระบุตัวตน
    \item ควรสำรวจพื้นที่ติดตั้งมอดูลกล้อง เนื่องจากต้องพิจารณาสภาพแสงของพื้นที่ติดตั้งเพื่อพัฒนาการปรับแต่งรูปภาพให้สว่าง หรือ มืด ซึ่งมีผลต่อความแม่นยำในการระบุตัวตน
    \item สามารถใช้กล้องแบบโฟกัสคงที่ เพื่อกำหนดระยะของการตรวจจับใบหน้าที่ชัดเจน ทำให้สามารถลดการความพร่ามัวของรูปภาพใบหน้าส่งผลให้ความแม่นยำในการระบุตัวตนสูงขึ้น 
\end{enumerate}
