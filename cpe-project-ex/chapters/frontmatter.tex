\maketitle
\makesignature

\ifproject
\begin{abstractTH}
% เขียนบทคัดย่อของโครงงานที่นี่

% การเขียนรายงานเป็นส่วนหนึ่งของการทำโครงงานวิศวกรรมคอมพิวเตอร์
% เพื่อทบทวนทฤษฎีที่เกี่ยวข้อง อธิบายขั้นตอนวิธีแก้ปัญหาเชิงวิศวกรรม และวิเคราะห์และสรุปผลการทดลองอุปกรณ์ และระบบต่าง ๆ
% \enskip อย่างไรก็ดี การสร้างรูปเล่มรายงานให้ถูกรูปแบบนั้นเป็นขั้นตอนที่ยุ่งยาก
% แม้ว่าจะมีต้นแบบสำหรับใช้ในโปรแกรม Microsoft Word แล้วก็ตาม
% แต่นักศึกษาส่วนใหญ่ยังคงค้นพบว่าการใช้งานมีความซับซ้อน และเกิดความผิดพลาดในการจัดรูปแบบ กำหนดเลขหัวข้อ และสร้างสารบัญอยู่
% \enskip ภาควิชาวิศวกรรมคอมพิวเตอร์จึงได้จัดทำต้นแบบรูปเล่มรายงานโดยใช้ระบบจัดเตรียมเอกสาร
% \LaTeX{} เพื่อช่วยให้นักศึกษาเขียนรายงานได้อย่างสะดวกและรวดเร็วมากยิ่งขึ้น

การระบุตัวตนด้วยการใช้รูปภาพใบหน้าบุคคลเพื่อเข้าสู่สถานที่เป็นหนึ่งในรูปแบบการยืนยันตัวตนที่ช่วยลดการแพร่ระบาดของโรคติดเชื้อไวรัสโคโรนา (COVID-19)
การใช้รูปภาพใบหน้าบุคคลนั้นยังมีจุดอ่อนเมื่อใบหน้าบุคคลมีการเปลี่ยนแปลง เช่น หนวดเครายาวขึ้น สวมแว่นตา เป็นต้น จึงมีการใช้กลวิธีการเรียนรู้แบบต่อเนื่องเพื่อนำรูปภาพใบหน้าใหม่ไปทำการเรียนรู้
เพื่อลดจุดอ่อนของการใช้รูปภาพใบหน้าบุคคล ทางเข้าสถานที่นั้นมีพื้นที่ในการติดตั้งน้อยในโครงงานนี้จึงเลือกใช้ Raspberry Pi และในโปรเจกต์นี้ได้ทำการติดตั้งที่ทางเข้าห้องแลป OASYS 
เนื่องจาก Raspberry Pi นั้นมีประสิทธิภาพไม่เพียงพอต่อการระบุตัวตน หรือ การเรียนรูปรูปภาพใบหน้าบุคคล ในโปรเจกต์นี้จึงใช้ Raspberry Pi ทำหน้าที่ตรวจจับใบหน้า และใช้เซิร์ฟเวอร์ในการระบุตัวตน 
และเรียนรู้รูปภาพใบหน้าบุคคล โดยแบบจำลองการตรวจจับใบหน้าบน Raspberry Pi นั้นสามารถที่จะตรวจจับใบหน้าบุคคลได้ดี ซึ่งเวลาเฉลี่ยในการแสดงผลการระบุตัวตนหลังจากการตรวจจับใบหน้านั้นน้อยกว่า 500 มิลลิวินาที 
และยังมีการยืนยันตัวตนด้วยรหัสอีกขึ้นตอนเพื่อลดความผิดพลาดในการระบุตัวตน ทำให้ผู้ใช้งานยังได้รับความสะดวก ปลอดภัยจากโรคติดเชื้อไวรัสโคโรนา (COVID-19) จากการใช้งาน และยังมีความแม่นยำ

\end{abstractTH}

\begin{abstract}
% The abstract would be placed here. It usually does not exceed 350 words
% long (not counting the heading), and must not take up more than one (1) page
% (even if fewer than 350 words long).

% Make sure your abstract sits inside the \texttt{abstract} environment.

Facial recognition technology is one form of identity verification that can help reduce the spread of COVID-19 by allowing individuals to 
access buildings without the need for physical contact. However, using facial images has its weaknesses when a person's face changes, 
such as by growing a beard or wearing glasses. To improve the accuracy of new facial images, adaptive machine learning algorithms are employed.

Due to limited space, this project utilized Raspberry Pi, which was installed at the entrance of the OASYS lab. 
Although Raspberry Pi's performance is insufficient for facial recognition and learning images, 
it was used to detect faces while a server was used for recognizing and training facial models. 
The facial detection model on the Raspberry Pi works well and takes less than 500 milliseconds on average to 
verify identity after detecting a face. Additionally, to reduce the likelihood of identity errors, there is a pin code 
confirmation process that makes it convenient and safe for users to use and helps prevent the spread of COVID-19.


\end{abstract}

\iffalse
\begin{dedication}
This document is dedicated to all Chiang Mai University students.

Dedication page is optional.
\end{dedication}
\fi % \iffalse

\begin{acknowledgments}
\indent โครงงานนี้จะไม่สำเร็จลุล่วงลงได้ ถ้าไม่ได้รับความกรุณาจาก ผศ.ดร.ภาสกร แช่มประเสริฐ
อาจารย์ที่ปรึกษา ที่ได้สละเวลาให้ความช่วยเหลือทั้งให้คำแนะนำ ให้ความรู้และแนวคิดต่าง ๆ รวมถึง
อ.ดร.\,ณัฐนันท์ พรหมสุข และ ผศ.ดร.กำพล วรดิษฐ์ ที่ให้คำปรึกษาจนทำให้โครงงานเล่มนี้เสร็จ
สมบูรณ์ไปได้ \\
\indent ขอบคุณห้องวิจัย OASYS ภาควิชาวิศวกรรมคอมพิวเตอร์ คณะวิศวกรรมศาสตร์
มหาวิทยาลัยเชียงใหม่ ที่เอื้อเฟื้อสถานที่ในการทำโครงงาน สนับสนุนอุปกรณ์ต่าง ๆ และ
ขอขอบคุณ นาย กมลพัฒน์ สุนทรพงศ์ และ นางสาวโชติโรส ประถม ที่คอยให้ความช่วยเหลือในการทำโครงงานมาโดยตลอด \\
\indent ขอขอบคุณทาง ITSC ที่ได้ให้ใช้เซิร์ฟเวอร์ และขอบคุณเพื่อน ๆ ที่ให้กำลังใจรวมถึงคำแนะนำที่ดีตลอดการทำโครงงานที่ผ่านมา 
รวมทั้งขอขอบพระคุณอีกหลาย ๆ ท่านที่ไม่ได้เอ่ยนามมา ณ ที่นี้ ที่ได้ให้ความ
ช่วยเหลือตลอดมา หากหนังสือโครงงานเล่มนี้มีข้อผิดพลาดประการใด กระผมขอน้อมรับด้วยความ
ยินดี


% \texttt{acknowledgment} environment.

\acksign{2022}{12}{12}
\end{acknowledgments}%
\fi % \ifproject

\contentspage

\ifproject
% \figurelistpage

% \tablelistpage
\fi % \ifproject

% \abbrlist % this page is optional

% \symlist % this page is optional

% \preface % this section is optional
