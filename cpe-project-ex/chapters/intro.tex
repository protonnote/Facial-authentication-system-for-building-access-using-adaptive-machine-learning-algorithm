\chapter{\ifenglish Introduction\else บทนำ\fi}

\section{\ifenglish Project rationale\else ที่มาของโครงงาน\fi}
การยืนยันตัวตนในการเข้าสถานที่หลายรูปแบบ เช่น การแสกนลายนิ้วมือ การใช้บัตรประจำตัว  การระบุเอกลักษณ์ด้วยคลื่นวิทยุ (Radio Frequency Identification: RFID) และอื่น ๆ อีกมากมาย 
ในปัจจุบันมีสถานการณ์โควิด-19 ยังมีการแพร่ระบาด ทำให้ผู้คนไม่สามารถพบปะระหว่างพนักงานต้อนรับกับผู้ที่เข้าสู่สถานที่ และการสัมผัสกับอุปกรณ์ยืนยันตัวตนทำให้เกิดการแพร่กระจากของโรค 
ส่งผลให้การยืนยันตัวตนในการเข้าสถานที่โดยใช้รูปถ่ายใบหน้าจะช่วยลดการแพร่ระบาดของเชื้อโรค และมีความสะดวกในการใช้งานไม่ต่างกับการยืนยันตัวตนแบบอื่น แต่เมื่อใบหน้าของบุคคลมีการเปลี่ยนแปลงตลอดในทุกวัน 
เช่น มีหนวด ไม่มีหนวด ผมสั้น ผมยาว ใส่แว่น ไม่ใส่แว่น มีผลทำให้การระบุตัวตนด้วยการใช้ภาพใบหน้าจะเกิดความผิดพลาด ซึ่งเป็นปัญหาหลักของการระบุตัวตนด้วยการใช้ภาพใบหน้า
ในสถาณะการณ์ปัจจุบันเกิดวิกฤติขาดแคลนชิปส่งผลให้อุปกรณ์อิเล็กทรอนิกส์มีราคาสูงขึ้น และทางเข้าสถานที่มีพื้นที่จำกัด
% จึงเกิดเป็นที่มาของโครงงานนี้ โดยมีการออกแบบอุปกรณการนำรูปภาพใบหน้าที่ได้รับเข้ามาใหม่ในแต่ละครั้งของการระบุตัวตนนั้น 
% ไปทำการเรียนรู้ใบหน้าใหม่ ทำให้ระบบสามารถจดจำภาพใบหน้าใหม่ที่มีการเปลี่ยนแปลง 
% และความแม่นยำในการระบุตัวตนจะสูงขึ้นเมื่อมีจำนวนรูปภาพใบหน้าที่เก็บไว้มากขึ้น และหลายหลายรูปแบบ ส่งผลให้สามารถแก้ปัญหาของการเปลี่ยนแปลงทางใบหน้าได้ 

\indent จึงเป็นที่มาของโครงงาน โดยได้ออกแบบอุปกรณ์ตรวจจับใบหน้า และแสดงผล ให้มีขนาดที่เล็ก น้ำหนักเบา ใช้เวลาในการติดตั้งที่สั้น ใช้พื้นที่ในการติดตั้งที่น้อย และงบประมาณของระบบที่ใช้นั้นไม่สูง 
โดยระบบตรวจจับใบหน้าที่ได้ออกแบบนั้นสามารถที่จะแยกแยะใบหน้ามนุษย์ได้ทั้งการใส่หน้ากากอนามัยหรือไม่ใส่ และออกแบบเซิร์ฟเวอร์ให้ทำหน้าที่ระบุตัวตน และเรียนรู้รูปภาพใบหน้าเพื่อนำไปเป็นแบบจำลองใบหน้าบุคคล 
ซึ่งจะนำรูปภาพใบหน้าที่ได้รับการยืนยันตัวตนในครั้งใหม่ไปทำการเรียนรู้ด้วย เมื่อมีการระบบตัวตนผิดพลาดหรือมีความแม่นยำที่ต่ำกว่าค่าที่กำหนดก็จะมีการใช้รหัสในการระบุตัวตน 
เพื่อเข้ามาช่วยในการยืนยันตัวตนให้มีความปลอดภัยมากยิ่งขึ้น
% โดยเมื่อระบบตัวตนแล้วนั้นผู้ใช้จะสามารถให้คะแนนของการระบุตัวตนเพื่อที่จะเป็นการยืนยันการระบุตัวตนว่ามีความถูกหรือผิด

\section{\ifenglish Objectives\else วัตถุประสงค์ของโครงงาน\fi}
\begin{enumerate}
    \item เพื่อพัฒนาระบบตรวจจับใบหน้ามนุษย์ให้มีความแม่นยำที่สูง
    \item เพื่อให้ระบบมีความเหมาะสมของอุปกรณ์ที่ติดตั้ง
    \item เพื่อให้ระบบสามารถนำไปใช้งานได้จริง
\end{enumerate}

\section{\ifenglish Project scope\else ขอบเขตของโครงงาน\fi}
โดยระบบตรวจจับใบหน้าจะทำการติดตั้งหน้าทางเข้าห้องกลุ่มวิจัยทฤษฎีและการประยุกต์ใช้การหาค่าที่เหมาะสมที่สุดในระบบทางวิศวกรรม 
(OASYS Research Group Optimization Theory and Applications for Engineering SYStems Research Group: OASYS) 
และปรับให้มีความแม่นยำมากที่สุดให้ยังคงความพึงพอใจของผู้ใช้ห้องได้

\subsection{\ifenglish Hardware scope\else ขอบเขตด้านฮาร์ดแวร์\fi}
\begin{enumerate}
    \item ระบบจะสามารถค้นหาใบหน้าได้จะต้องมีพื้นที่ที่มีแสงสว่างเพียงพอ
    \item พื้นที่ที่ทำการติดตั้งต้องมีสัญญานอินเทอร์เน็ตทั้งไร้สายหรือผ่านสายแลน (ข่ายงานบริเวณเฉพาะที่)
    \item พื้นที่ที่ทำการติดตั้งต้องไม่มีผุ้คนพลุกพล่าน
    \item โปรแกรมการเรียนรู้ของเครื่องที่ไม่เกินกําลังด้านฮาร์ดแวร์ของเซิร์ฟเวอร์ที่ใช้เรียนรู้แบบจำลอง (model)
\end{enumerate}

\subsection{\ifenglish Software scope\else ขอบเขตด้านซอฟต์แวร์\fi}
\begin{enumerate}
    \item สามารถจัดเก็บข้อมูลและรูปภาพใบหน้าได้
    \item สามารถที่จะเรียนรู้รูปภาพใหม่ ที่เข้ามาจัดเก็บได้
    \item ระบบใช้เวลาในการตรวจจับใบหน้า ส่งภาพไปหน้าไปยังเซิร์ฟเวอร์ ระบุตัวตน และส่งผลลัพธ์กลับมาแสดงจะให้เวลาไม่เกิน 20 วินาที
\end{enumerate}

\section{\ifenglish Expected outcomes\else ประโยชน์ที่ได้รับ\fi}
\begin{enumerate}
    \item ผู้ที่เข้าสู่สถานที่ลดความเสี่ยงที่จะได้รับเชื่อโรค
    \item ระบบสามารถที่จะระบุตัวตนในเวลาที่น้อย ทำให้ยังคงความสะดวกในการเข้าสู่สถานที่ได้ไม่ต่างจากการเข้าสู่สถานที่รูปแบบอื่น ๆ
    \item ระบบสามารถส่งต่อสัญญานหรือข้อมูลไปยังส่วนอื่น ๆ ได้ เช่น บอกทางไปห้องทำงาน เปิดเครื่องคอมพิวเตอร์ในห้องทำงาน บอกตารางงานของบุคคลนั้น และจดจำเวลาเข้างานหรือออกงาน เป็นต้น
\end{enumerate}

\section{\ifenglish Technology and tools\else เทคโนโลยีและเครื่องมือที่ใช้\fi}
\subsection{\ifenglish Hardware technology\else เทคโนโลยีด้านฮาร์ดแวร์\fi}
\begin{enumerate}
    \item คอมพิวเตอร์ขนาดเล็ก (Raspberry Pi 4 Model B)
    \item กล้องเว็บแคมส์ที่ใช้การเชื่อมต่อผ่านช่องยูเอสบี (บัสคอมพิวเตอร์) (webcam)
    \item จอภาพ (monitor)
    \item แผงแป้นอักขระ (keyboard)
    \item เครื่องบริการ (server)
\end{enumerate}

\subsection{\ifenglish Software technology\else เทคโนโลยีด้านซอฟต์แวร์\fi}
\begin{enumerate}
    \item Python : ภาษาที่ใช้ในการค้นหาภาพใบหน้าบุคคล การส่งรูปภาพใบหน้า และการทำเว็ปเซิร์ฟเวอร์สำหรับรับรูปภาพ
    \item OpenCV : ไลบรารี (Library) ใช้ในการค้นหาใบหน้าบุคคล และใช้ในการระบุตัวตน
    % \item TensorFlow : ไลบรารี (Library) สำหรับการเรียนรู้รูปภาพใบหน้าออกมาเป็นโมเดลโดยสามารถใช้งานได้ดีกับภาษา Python
    \item Tkinter : ไลบรารี (Library) สำหรับการพัฒนา (Graphical User Interface: GUI) ที่ใช้ภาษาไพธอน (Python)
    \item Open Face : โมเดลที่ใช้ในการระบุตันตนบุคคล
    \item MediaPipe : ไลบรารี (Library) ของ (Machine Learning: ML) หรือ (Deep Learning: DL) ที่พัฒนาโดย Google ใช้ในการตรวจจับใบหน้าบุคคล
    \item Flask Framework : เป็นโครงสร้างของ Restful API ที่ใช้ในการทำเว็ปเซิร์ฟเวอร์ที่รับรูปภาพโดยเป็นภาษาไพธอน (Python) ทำให้สามารถเรียกใช้งาน OpenCV หรือ TensorFlow เมื่อรับรูปภาพสำเร็จและส่งผลลัพธ์
    \item Rest API : ใช้ในการสร้างเว็ปเซิร์ฟเวอร์สำหรับรับรูปภาพบนเซิร์ฟเวอร์
    \item Application Programming Interface : ใช้ในการส่งรูปภาพผ่านเอชทีทีพี (HyperText Transfer Protocol: HTTP)
    \item Virtual Studio Code : ใช้ในการพัฒนาการค้นหาใบหน้าแบบเรียลไทม์และทำเว็ปเซิร์ฟเวอร์สำหรับรับรูปภาพ
    
\end{enumerate}

\section{\ifenglish Project plan\else แผนการดำเนินงาน\fi}

\begin{plan}{12}{2021}{3}{2023}
    \planitem{12}{2021}{1}{2022}{ศึกษา และการตรวจจับใบหน้า และการทำงานบน raspbian os และการบีบอัดไฟล์รูปภาพ}
    \planitem{1}{2022}{2}{2022}{ศึกษา และทดลองการส่งรูปภาพผ่าน RESTful API และ Python Flask framework และเทคนิคการปรับรูปภาพ}
    \planitem{3}{2022}{3}{2022}{ศึกษา และทดลองการทำงานของ DNN และการเรียนรู้ภาพใบหน้าบุคคล หรือ Train model }
    \planitem{10}{2022}{11}{2022}{เก็บข้อมูลรูปภาพใบหน้าผู้ใช้งานห้องวิจัย OASYS และออกแบบให้ระบบสามารถตรวจจับใบหน้าได้ดีขึ้น และส่งภาพใบหน้าเร็วขึ้น}
    \planitem{11}{2022}{12}{2022}{ติดตั้ง และทดสอบระบบ และออกแบบและพัฒนาโมเดลการเรียนรู้ภาพใบหน้าบุคคลบนเซิร์ฟเวอร์ และการส่งผลลัพธ์}
    \planitem{12}{2022}{1}{2023}{ออกแบบ และพัฒนา GUI ตอบรับผลลัพธ์ ส่งผลลัพธ์ไปยังเซิร์ฟเวอร์ และเซิร์ฟเวอร์จัดการกับผลลัพธ์ที่ได้รับกลับมา}
    \planitem{1}{2023}{2}{2023}{ทดสอบทั้งระบบ ปรับปรุงระบบ และปรับแต่งระบบให้มีประสิทธิภาพขึ้น}
    \planitem{2}{2023}{3}{2023}{เขียนรายงานสรุปผลการทำงาน}
\end{plan}

\section{\ifenglish Roles and responsibilities\else บทบาทและความรับผิดชอบ\fi}
รับผิดชอบทุกส่วนของโครงงานนี้ โดยที่ต้องใช้ความรู้ด้าน Computer vision, Web service, Storage, \\ Rest API, Machine Learning และพัฒนาการเรียนรู้รูปภาพใบหน้า

\section{\ifenglish%
Impacts of this project on society, health, safety, legal, and cultural issues
\else%
ผลกระทบด้านสังคม สุขภาพ ความปลอดภัย กฎหมาย และวัฒนธรรม
\fi}

สามารถช่วยลดการแพร่ระบาดของเชื้อโควิด-19 ของพนักงานในสถานที่ มีการเก็บรูปภาพบุคคลที่เข้าสถานที่โดยเมื่อมีเหตุการณ์ก็นำรูปที่บันทึกมาใช้เป็นหลักฐานได้โดยรูปภาพใบหน้านั้นจะไม่อนุญาติให้ผู้อื่นนำไปใช้ได้จะสามารถใช้ได้ก็ต่อเมื่อมีการขออณุญาติเรียบร้อยซึ่งจะไม่ขัดกับกฎหมาย 
รูปภาพที่ส่งไปให้เซิร์ฟเวอร์นั้นมีการเข้ารหัสเพื่อป้องกันการโจรกรรมได้ เมื่อยืนยันตัวตนสำเร็จก็สามารถนำข้อมูลหรือสัญญานไปยังระบบอื่น ๆ ได้ เช่นระบบบันทึกการเข้างาน ระบบบอกทางไปยังห้องทำงาน เป็นต้น ทำให้เป็นอีกช่องทางในการยืนยันตัวตนเพื่อเข้าสู่สถานที่
